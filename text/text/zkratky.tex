\cleardoublepage
\chapter*{\listofabbrevname}
\phantomsection
\addcontentsline{toc}{chapter}{\listofabbrevname}

\begin{acronym}[KolikMista]
%	\acro{A}
%		{Ampér (základní jednotka proudu)}
	\acro{Bd}
		{Baud - jednotka modulační rychlosti}
	\acro{C}
		{Kapacita}
	\acro{COM}
		{Common - sériový port RS-232}
	\acro{DPS}		
		{Deska plošného spoje}
	\acro{FLASH}
		{Typ paměti, která je trvalá - nesmaže se ani při ztrátě napájení}
	\acro{GND}
		{Ground - pin, který má nulový potenciál, vůči němu jsou referencované všechny ostatní signály}
	\acro{GPIO}
		{General Purpoise Input Output - piny, které mohou být vstupní nebo výstupní}
	\acro{kB}
		{Kilobajt - jednotka velikosti paměti}
%	\acro{}[k$\Omega$]
%		{Kiloohm (jednotka odporu)}	
	\acro{LDO}
		{Low-dropout regulator - regulátor napětí s nízkým úbytkem}
	\acro{LED}
		{Light-Emitting Diode - dioda emitující světlo}
	\acro{Li-Ion}
		{Lithium-iontový akumulátor - druh nabíjecí baterie}
%	\acro{mA}
%		{Miliampér (jednotka proudu)}
	\acro{mAh}
		{Miliampérhodina (jednotka kapacity používaná hlavně u~baterií)}
	\acro{MB}
		{Megabajt - jednotka velikosti paměti}
	\acro{MOSFET}
		{Metal Oxide Semiconductor Field Effect Transistor - tranzistor řízený elektrickým polem}	
	\acro{M3}
		{Metrický závit o~průměru 3~mm}
	\acro{RS-232}
		{Druh sériového komunikačního rozhraní} 
	\acro{RX}
		{Reciever - přijímač sériového rozhraní}
	\acro{SMD}
		{Surface Mount Device - součástky určené pro povrchovou montáž} 
	\acro{SPI}
		{Serial peripheral interface - sériové periferní komunikační rozhraní}
	\acro{SRAM}
		{Static Random Acess Mamory - rychlá statické paměť, která se smaže při ztrátě napájení}	
	\acro{THT}
		{Through-hole technology - vývodová technologie součástek}
	\acro{TX}
		{Transciever - vysílač sériového rozhraní} 
	\acro{USB}
		{Universal Serial Bus - univerzální sériová sběrnice, která se používá pro připojení zařízení k~počítači}
%	\acro{V}
%		{Volt (základní jednotka napětí)}
	\acro{VDD}
		{Označení napájecího napětí}

\end{acronym}
