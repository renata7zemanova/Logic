\chapter*{Úvod}
\phantomsection
\addcontentsline{toc}{chapter}{Úvod}

Tato práce se zabývá návrhem elektronické hry Logic. Tato hra by se měla co nejvíce podobat deskové hře a~musí respektovat pravidla deskové 
hry. Hra Logic je určená pro všechny věkové kategorie hráčů. Podporuje a~rozvíjí logické myšlení. 

Hra musí být navržena jako přenosné zařízení, takže musí být kladen důraz na spotřebu jednotlivých použitých komponentů.

Hra Logic není komerčně dostupná v~elektronické verzi. Existuje pouze ve verzi deskové hry pro dva hráče. Elektronická hra Logic je 
navržena pro jednoho hráče. Hra je jedinečná svými ovládacími prvky a~způsobem hry. 

V~této práci je popsán návrh elektronické hry. Na začátku jsou popsána pravidla deskové hry, která jsou respektována i~u~elektronické hry. 
Elektronická hra Logic je tvořena DPS, která obsahuje veškeré potřebné komponenty pro funkci hry i~pro její ovládání.

Na závěr je popsán způsob elektronické hry a~její ovládání.


Úvod studentské práce, např\,\dots

Tato práce se věnuje oblasti \acs{DSP} (\acl{DSP}), zejména jevům, které nastanou při nedodržení Nyquistovy podmínky pro \ac{symfvz}.%
\footnote{Tato věta je pouze ukázkou použití příkazů pro sazbu zkratek.}

Šablona je nastavena na \emph{dvoustranný tisk}. Pokud máte nějaký závažný důvod sázet (a~zejména tisknout) jednostranně, nezapomeňte si přepnout volbu \texttt{twoside} na \texttt{oneside}!