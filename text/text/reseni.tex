\chapter{Pravidla hry}
Hra Logic je desková hra pro dva hráče \cite{Logic_pravidla}. Jeden hráč určí hledanou kombinaci, dále bude označován jako 
hráč A, a druhý tuto kombinaci za pomocí logických úvah a
vyhodnocení hráčem A hledá, dále bude označován jako hráč B.

Hráči si určí herní pozice. Hráč A vybere barvy a následně je v určitém pořadí vloží do zadávacího pole a zakryje je, aby tuto
kombinaci spoluhráč neviděl. Hráč B je v této chvíli otočen. Hráč B následně zvolí náhodnou kombinaci barev a jejich pozic. Po 
ukončení tahu nechá
hráče A, aby tah vyhodnotil. Hráč A vyhodnotí tah následujícím způsobem. Pokud hráč B vložil správnou barvu na správnou pozici, tak vloží 
do vyhodnocovací sekce černý kolík. Pokud vložil barvu, která se v zadání vyskytuje, ale vložil ji na nesprávnou pozici, tak vloží bílý kolík.
Pokud zůstanou některé pozice neobsazené, tak to znamená, že daná barva se v zadání nevyskytuje. Poté začne hráč B na základě vyhodnocení a 
svých všech předchozích tahů hledat správnou kombinaci.

Cílem hry je nalézt správnou kombinaci dříve, než skončí plocha herního pole.

Hru lze hrát ve více variantách. Hráči se mohou domluvit, zda zadání může, nebo nesmí obsahovat volnou pozici. Zároveň existuje 
více variací této hry. Většinou se liší v délce hledané kombinace.

\chapter{Popis zapojení}

\begin{figure}[!h]
  \begin{center}
    \includegraphics[scale=0.6]{obrazky/v0_blokove_schema.jpg}
  \end{center}
  \caption[Blokové schéma zapojení v0.0]{Blokové schéma zapojení v0.0.}
\end{figure}

\begin{figure}[!h]
  \begin{center}
    \includegraphics[scale=0.6]{obrazky/v1_blokove_schema.jpg}
  \end{center}
  \caption[Blokové schéma zapojení v1.0]{Blokové schéma zapojení v1.0.}
\end{figure}

\section{Procesor}
Byl vybrán procesor ESP32-PICO-D4.

Tento procesor obsahuje \cite{PICO_datasheet}: %napsáno zatím pro WROOM - opravit
\begin{itemize}
    \item WiFi,
    \item Bluetooth,
    \item procesor,
    \item 32 GPIO pinů,
    \item dvoujádrový 32bitový procesor Xtensa LX6,
    \item 520 kB SRAM, %?
    \item 16 MB FLASH. %?
  \end{itemize}

  Napájecí napětí tohoto procesoru je od 3,0 do 3,6 V a průměrný odběr proudu je 80 mA \cite{PICO_datasheet}. ESP32-PICO má vyvedeno 32 GPIO pinů, které je 
  možno softwarově nastavit jako vstupní nebo výstupní. Na tyto piny lze poté připojit různá zařízení. Vstupním senzorem může 
  být typicky tlačítko a výstupním indikátorem např. LED. Tato zařízení zprostředkovávají komunikaci mezi procesorem a okolním 
  světem.

  \begin{figure}[!h]
    \begin{center}
      \includegraphics[scale=1]{obrazky/ESP32_PICO_schema.png}
    \end{center}
    \caption[Schéma zapojení procesoru ESP32-PICO]{Schéma zapojení procesoru ESP32-PICO \cite{PICO_datasheet}.}
  \end{figure}

  GPIO piny IO16 a IO17 nemohou být použity, protože ESP32-PICO má na těchto pinech připojenou flash paměť \cite{PICO_datasheet}.
  Pokud by na tento pin bylo připojeno nějaké zařízení, tak by procesor ztratil přístup ke své paměti.

  GPIO piny IO34 a vyšší jsou pouze vstupní \cite{PICO_datasheet}. Vstupní piny nemají softwarově zapojitelný pullup rezistor. 
  Pokud je tedy zapotřebí pullup rezistor, musí se fyzicky zapojit.
  
  \section{Napájení}
  \subsection{v0.0}
  \subsection{v1.0}
  Ve verzi 1.0 byl změněn způsob napájení. Napájení v této verzi neprobíhá přes baterie, ale pouze přes USB konektor, přes 
  který ve verzi 0.0 probíhalo pouze programování procesoru a nabíjení baterií. 

  Výhody
  \begin{itemize}
    \item absence nabíjecího obvodu,
    \item absence hlídání stavu nabití baterie,
    \item absence ochrany proti přepólování (USB konektor je uzpůsoben svým tvarem, aby uživatel nemohl napájení přepólovat.),
    \item napájení inteligentních LED napětím přímo z USB (Inteligentní LED mají menší odběr proudu.).
  \end{itemize}

  Nevýhody
  \begin{itemize}
    \item předpokládá se, že uživatel vlastní powerbanku.% vymyslet další
  \end{itemize}

  Byl zvolen konektor USB Micro, protože se jedná o nejrozšířenější USB konektor dnešní doby.

  \section{Stepdown}
  Procesor ESP32-PICO-D4 má napájecí napětí 3,3 V. Napětí z powerbanky přes USB je 5 V. Proto je tedy zapotřebí zapojit 
  stepdown, který bude vytvářet z napájecího napětí 5 V napájecí napětí pro procesor.

  Jako stepdown byl zvolen čip SY8105.

  \begin{table}[!h]
    \caption{Parametry čipu SY8105 \cite{SY8105_datasheet}}
    \begin{center}
        \begin{tabular}{|c|c|}
            \hline
            Vstupní napětí             & 1,5 - 18 V \\
            \hline
            Maximální výstupní proud   & 5 A \\
            \hline
        \end{tabular}    
    \end{center}
\end{table}

  \begin{figure}[!h]
    \begin{center}
      \includegraphics[scale=0.4]{obrazky/SY8105_schema.png}
    \end{center}
    \caption[Schéma zapojení čipu SY8105]{Schéma zapojení čipu SY8105 \cite{SY8105_datasheet}.}
  \end{figure}

  Výstupní napětí je závislé na poměru odporů R11 a R12. Nastavené velikosti odporů jsou pro výstupní napětí 3,3 V pro procesor 
  ESP32-PICO.

  \section{Převodník z USB na RS-232}
  Procesor ESP32-PICO používá jako komunikační rozhraní linku RS-232. Programování ale probíhá přes USB, které toto rozhraní
  nemá. Proto bylo potřeba použít převodník z USB na rozhraní RS-232.
  
  Pro převod z USB na RS-232 byl použit čip CP2102, který  zároveň převádí logiku z 0 - 5 V na logiku 0 - 3,3 V 
  \cite{CP2102_datasheet}. Zapojení čipu bylo převzato z kitu ESP32-DEVKITC, kde je zapojení funkční s použitím ESP32-WROOM.
  ESP32-PICO a ESP32-WROOM se liší pouze v drobnostech, proto mohlo být toto zapojení převzato.

  Čip CP2102 dokáže komunikovat vekým množstvím komunikačních rychlostí (300, 9600, 19200, 38400, 115200, 256000, atd) 
  \cite{CP2102_datasheet}. PC započne komunikaci určitou rychlostí a tento čip tuto komunikaci zachytí, určí rychlost 
  a touto rychlostí začne probíhat programování.
  %přidat
  \begin{table}[!h]
    \caption{Parametry čipu CP2102 \cite{CP2102_datasheet}}
    \begin{center}
        \begin{tabular}{|c|c|}
            \hline
            Typický odběr proudu   & 9,5 mA \\
            \hline
        \end{tabular}    
    \end{center}
\end{table}

\begin{figure}[!h]
    \begin{center}
      \includegraphics[scale=0.6]{obrazky/CP2102_schema.png}
    \end{center}
    \caption[Schéma zapojení převodníku z USB na RS-232]{Schéma zapojení převodníku z USB na RS-232 \cite{Devkit_schema}.}
\end{figure}

Z USB jsou signály D+ a D- připojeny k čipu CP2102. Tento čip signál z USB převede na signály RX a TX, které mají výstup 
na pinech RXD a TXD. Následně jsou tyto signály připojeny k procesoru ESP32-PICO. Signály RX a TX musí být překříženy – RX 
CP2102 je připojeno na TX ESP32-PICO a TX CP2102 je připojeno na RX ESP32-PICO. 

LED D10 a D11 slouží k indikaci komunikace s procesorem. Jsou zapojeny podle datasheetu \cite{CP2102_datasheet}. Pokud je do 
procesoru nahráván program, tak LED D10 a D11 blikají.

\begin{figure}[!h]
    \begin{center}
      \includegraphics[scale=0.5]{obrazky/CP2102_LED.png}
    \end{center}
    \caption[Zapojení LED pro indikaci komunikace čipu CP2102 s procesorem]{Zapojení LED pro indikaci komunikace čipu CP2102 s procesorem \cite{CP2102_datasheet}.}
\end{figure}

\section{Inteligentní LED}
Byly vybrány inteligentní LED typu WS2812C. Tento typ inteligentních LED je určen pro přenosná zařízení, díky jejich nízké 
spotřebě. Tyto inteligentní LED jsou plně kompatibilní s typem WS2812B \cite{WS2812C_datasheet}. Ktěmto inteligentním LED 
existují knihovny, které usnadňují softwarovou práci s nimi.

Každá inteligentní LED má v sobě procesor, který slouží pro zpracování dat. 
Inteligentní LED WS2812C se zapojují za sebou přes piny DATA IN a DATA OUT. Každá inteligentní LED převezme data z pinu DATA IN, která jsou 
pro ni, a zbytek pošle ven přes pin DATA OUT.

\begin{figure}[!h]
    \begin{center}
      \includegraphics[scale=0.5]{obrazky/WS2812C_spojeni.png}
    \end{center}
    \caption[Zapojení inteligentních LED WS2812C]{Zapojení inteligentních LED WS2812C \cite{WS2812C_datasheet}.}
\end{figure}

\begin{table}[!h]
  \caption{Parametry inteligentních LED WS2812C \cite{WS2812C_datasheet}}
  \begin{center}
      \begin{tabular}{|c|c|}
          \hline
          Napájecí napětí   & 3,5 - 5,3 V \\
          \hline
          Výstupní napětí   & (VDD - 0,5) - (VDD + 0,5) V \\
          \hline
          Typický odběr proudu & 5 mA \\
          \hline
          Klidový odběr proudu & 0,3 mA \\
          \hline
      \end{tabular}    
  \end{center}
\end{table}

Ke každé inteligentní LED je připojen na napájení filtrační kondenzátor \cite{WS2812C_datasheet}, aby LED svítili kontinuálně 
a nedostal se jim na napájení žádný šum.

\subsection{Rozdělení}
Inteligentní LED jsou rozděneny do tří skupin. Skupina inteligentních LED pro zadání, skupina inteligentních LED pro herní pole 
a skupina inteligentních LED pro zobrazení vyhodnocení tahu.
Skupina inteligentních LED pro zadání obsahuje 4 LED a skupiny pro herní pole a pro vyhodnocení každá 40 LED.

\subsection{Zapínání napájení}
DPS je navrhována pro přenosnou aplikaci, a proto je potřeba zajistit její co nejnižší odběr. 

Inteligentní LED WS2812C mají spotřebu 0,3 mA, i když zrovna nesvítí žádnou barvou. Proto je herní pole dohromady s 
vyhodnocovacími LED rozděleno na 3 části. Do první části patří inteligentní LED se zadáním a první 4 čtveřice inteligentních LED 
z herního pole a z vyhodnocení. Do druhé části patří další 3 čtveřice inteligentních LED z herního pole a z vyhodnocení. 
Do třetí části patří poslední 3 čtveřice inteligentních LED z herního pole a z vyhodnocení.

Těmto 3 částem je postupně zapínáno napájecí napětí. Každé části se zapne napájení až pokud se hráč dostane do fáze, kdy 
danou oblast potřebuje. K zapínání dochází softwarově spínáním GPIO pinem procesoru.

Ke spínání slouží obvody s MOSFET tranzistory. MOSFET tranzistory byly zvoleny pro jejich nulovou spotřebu, narozdíl od 
bipolárních tranzistorů. 

\subsubsection{Popis funkce zapínání napájení}

\begin{figure}[!h]
  \begin{center}
    \includegraphics[scale=0.8]{obrazky/Zapinani_napajeni_LED.png}
  \end{center}
  \caption[Obvod pro zapínání napájení pro inteligentní LED]{Obvod pro zapínání napájení pro inteligentní LED.}
\end{figure}

Napájení inteligentních LED nelze spínat pouze jedním tranzistorem, protože logická 1 procesoru má hodnotu 3,3 V, ale je 
zapotřebí spínat 5 V. Pokud by byl pro spínání použit pouze jeden MOSFET tranzistor, tak by napájení bylo vždy sepnuto.

GPIO pin procesoru je nastaven do logické 0, dokud není potřebné přivedení napájecího napětí dané skupině. 
Při logické nule na gate 
tranzistoru Q4 je tranzistor zavřený. Tranzistor Q5 v tomto okamžiku drží zavřený rezistor R15.

Když je GPIO pin procesoru přepnut do logické 1, tak se tranzistor Q4 otevře. Otevřením tranzistoru Q4 je gate tranzistoru Q5 
připojen ke GND a tím se otevře i tranzistor Q5, kterým je sepnuto napájecí napětí dané skupině inteligentních LED.

Rezistor R14 udržuje tranzistor Q4 zavřený při nestandardních stavech pinu procesoru, jako je např. při resetu procesoru.

\section{Spínací prvky}
Přepínač SW1 slouží pro zapínání celé DPS. Tento přepínač připojuje napájecí napětí 5 V z USB k celému zbytku DPS.

Tlačítka slouží pro ovládání hry. Ke každému tlačítku je připojen kondenzátor o hodnotě 100 nF. Tento kondenzátor 
slouží pro filtraci zákmitů při zmáčknutí tlačítka. Filtrace se proto nemusí řešit softwarově.

\begin{figure}[!h]
  \begin{center}
    \includegraphics[scale=0.8]{obrazky/Tlacitka_zapojeni.png}
  \end{center}
  \caption[Zapojení tlačítek]{Zapojení tlačítek.}
\end{figure}

\section{PowerLED}
LED D1 a D2 slouží pro indikaci přítomnosti napájecího napětí.  LED D1 indikuje přítomnost napájecího napětí 
5 V a LED D2 indikuje přítomnost napájecího napětí 3,3 V.

\begin{figure}[!h]
  \begin{center}
    \includegraphics[scale=0.5]{obrazky/powerLED.png}
  \end{center}
  \caption[Zapojení LED pro indikaci napájecího napětí]{Zapojení LED pro indikaci napájecího napětí.}
\end{figure}

\chapter{Popis DPS}
DPS je navržena v programu KiCad a její parametry jsou určeny pro výrobu i osazení ve firmě JLCPCB \cite{JLCPCB}. Výrobní 
podklady proto musely být navrženy v souladu s jejich výrobními možnostmi \cite{JLCPCB_Capabilities}.

DPS má 4 vrstvy. Vnitřní vrstvy slouží pro napájení a vnější pro signálové dráhy. V jedné vnitřní vrstvě je po celé její ploše 
polygon GND a ve druhé vnitřní vrstvě jsou polyogny jednotlivých napájecích napětí.

Na vrchní straně jsou umístěny plošky pro osazení SMD součástek, protože firma JLCPCB osazuje pouze SMD součástky a pouze z jedné 
strany. THT součástky jsou připraveny na ruční pájení.

Signálové dráhy jsou vedeny tenkou dráhou a napájecí dráhy jsou vedeny širší dráhou. V signálových drahách tečou zanedbatelné 
proudy, proto mohou být co nejtenčí. Výrobce umožňuje vyrobit nejtenčí dráhu u čtyřvrstvé DPS 0,09 mm \cite{JLCPCB_Capabilities}. 
Aby nebyly použity krajní hodnoty, byla zvolena šířka signálové dráhy 0,150 mm.

Kondenzátoty u procesoru ESP32-PICO a u čipu CP2102 musí být umístěny co nejblíže jejich pouzdru. Tyto kondenzátory slouží pro 
filtraci šumu na napájení.

Dráhy od USB k čipu CP2102 D+ a D- fungují jako diferenciální pár. Proto musejí být jejich dráhy vedeny vedle sebe a blízko u 
sebe.

Rozložení součástek k čipu SY8105 na DPS může velmi ovlivnit jeho funkčnost. Rozložení a zapojení stepdownu 
bylo převzato z datasheetu.

\begin{figure}[!h]
  \begin{center}
    \includegraphics[scale=1]{obrazky/SY8105_rozlozeni_na_DPS.png}
  \end{center}
  \caption[Rozložení součástek kolem čipu SY8105 na DPS]{Rozložení součástek kolem čipu SY8105 na DPS \cite{SY8105_datasheet}.}
\end{figure}

Inteligentní LED WS2812C jsou rozděleny do 3 skupin, aby se hra co nejvíce podobala deskové hře. Inteligentní LED pro zadání 
jsou v horní části DPS. V levém sloupci pod zadáním jsou inteligentní LED, které slouží jako herní pole, a v pravém sloupci 
jsou inteligentní LED pro vyhodnocení tahu. Každá inteligentní LED musí mít svůj filtrační kondenzátor na napájení co nejblíže 
svému pouzdru.

\chapter{Oživení DPS}
\section{v0.0}
%Po dodání DPS z výroby byly zapájeny THT komponenty (pouzdro na baterii, USB konektor, tlačítka a vypínač). Po zapojení baterií
%18650 do pouzdra se rozsvítí zelená LED D1, která indikuje přítomnost napájecího napětí. Po připojení k počítači přes USB
%se rozsvítí LED D9. Ta signalizuje nabíjení baterií. Pokud se rozsvítí LED D8, znamená to, že baterie jsou nabité.

\section{v1.0}
DPS přijde z výroby ve stavu, kdy jsou osazeny pouze SMD komponenty. % fotka DPS

Poté je nutné ručně osadit THT součástky, tj. vypínač, tlačítka a konektor USB micro. Připojení DPS přes USB k powerbance, 
nebo do počítače, se rozsvítí LED D1 a D2, které indikují přítomnost napájecího napětí. LED D2 zároveň značí, že stepdown je 
funkční.

\chapter{Způsob ovládání elektronické hry}
První verze je navržena jako hra pro jednoho hráče. Funkci druhého hráče nahrazuje procesor.

Po zapnutí DPS stikneme tlačítko "Nová hra". V této chvíli se vygeneruje zadání, které není vidět, a první herní LED se 
rozbliká. Rozblikání herní LED značí pozici kurzoru. 
Kurzorem lze pohybovat pomocí tlačítek "Šipka vpravo" a "Šipka vlevo". Barvy herních LED se nastavují tlačítky ve spodní části 
DPS. Tato tlačítka jsou označena danými barvami.
Po ukončení tahu stiskneme tlačítko "Potvrdit tah". Proběhne vyhodnocení a zobrazí se na vyhodnocovacích LED. Kurzor se posune
na první LED v dalším řádku.
Po zadání správné kombinace barev a jejich pozic se rozsvítí zadání a hra je u konce. Pro novou hru stiskneme tlačítko
"Nová hra" a pro ukončení tlačítko "Konec".
Při stisku tlačítka "Konec" zhasnou všechny herní, vyhodnocovací LED i LED pro zadání. Poté je DPS připravena pro vypnutí
vypínačem.


















%možná vůbec nebude, nebo bude určitě jinde
%\section{Nabíjecí obvod}
%Pro nabíjecí obvod byl zvolen čip TP4056. Jeho zapojení bylo převzato z datasheetu [citace]. %tady bude obrázek schématu asi
%Velikost rezistoru $R_\textind{PROG}$ se volí podle nabíjecího proudu. 
%Tabulka rezistorů $R_\textind{PROG}$.. [citace]

%\begin{table}[!h]
%    \caption{Nastavení nabíjecího proudu rezitorem $R_\textind{PROG}$}
%    \begin{center}
%        \begin{tabular}{|c|c|}
%            \hline
%            $R_\textind{PROG}$ [kOhm] & Nabíjecí proud [mA] \\
%            \hline
%            10      & 130 \\
%            \hline
%            5       & 250 \\
%            \hline
%            4       & 300 \\
%            \hline
%            3       & 400 \\
%            \hline
%            2       & 580 \\
%            \hline
%            1,66    & 690 \\
%            \hline
%            1,5     & 780 \\
%            \hline
%            1,33    & 900 \\
%            \hline
%            1,2     & 1000 \\
%            \hline
%        \end{tabular}
%        
%    \end{center}
%\end{table}

%Nabíjení baterií by mělo probíhat při 0,5C, tudíž pro 18650 je to cca 0,5 A. Proto byl zvolen rezitor Rprog 2 kOhm.
