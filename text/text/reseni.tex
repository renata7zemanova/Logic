\chapter{Pravidla hry}
Logic je desková hra pro dva hráče. Jeden hráč určí hledanou kombinaci, dále bude označován jako hráč A, a druhý tuto kombinaci za pomocí logických úvah a
vyhodnocení hráčem A hledá, dále bude označován jako hráč B.

Hráči si určí herní pozice. Hráč A vybere barvy a následně je v určitém pořadí vloží do zadávacího pole a zakryje je, aby tuto
kombinaci spoluhráč neviděl. Hráč B je v této chvíli otočen. Hráč B následně zvolí náhodnou kombinaci barev a pozic. Po ukončení tahu nechá
hráče A, aby tah vyhodnotil. Hráč A vyhodnotí tah následujícím způsobem. Pokud hráč B vložil správnou barvu na správnou pozici, tak vloží 
do vyhodnocovací sekce černý kolík. Pokud vložil barvu, která se v zadání vyskytuje, ale vložil ji na nesprávnou pozici, tak vloží bílý kolík.
Pokud zůstanou některé pozice neobsazené, tak to znamená, že daná barva se v zadání nevyskytuje. Poté začne hráč B na základě vyhodnocení a 
svých všech předchozích tahů hledat kombinaci.

Cílem hry je nalézt správnou kombinaci dříve, než skončí plocha herního pole.

Těžší varianta spočívá v možnosti absence barvy - v zadání je mezera.

\chapter{Koncept}
Celá DPS je navržena tak, aby odpovídala svým vzhledem deskové hře. Ve spodní části jsou ovládací prvky.
Hráč pomocí tlačítek vybírá pozici a určuje jí barvu. Po ukončení tahu hra vyhodnotí zvolenou kombinaci a hráč může pokračovat dalším tahem.
Herní pole proto obsahuje sekci s herními LED a s vyhodnocovacími LED.

DPS je navržena pro strojní osazování u firmy JLCPCB.

\chapter{Použitý hardware}
\section{Procesor}
Byl zvolen procesor ESP32-PICO.
\section{LED}
Byly zvoleny programovatelné LED WS2812C. Tento typ LED je určen pro přenosná zařízení, díky jejich nízké spotřebě oproti typu WS2812B.
Tyto LED jsou řízeny programově. K procesoru jsou připojeny všechny jedním datovým pinem. Z procesoru jsou těmto LED předávána data za sebou v
RGB formátu. Každá LED má definovaný počet bytů, ve kterých jsou data právě pro jednu LED. Pokud LED registruje další data, tak je pošle dál.
Tímto způsobem se všechna data dostanou ke všem LED. V programu musí být tedy zachováno pořadí LED. 

Porovnání WS2812B s WS2812C

\chapter{Návrh DPS}
DPS je navržena v programu KiCad.

\section{Napájení}
Byly zvoleny baterie 18650.

\section{Nabíjecí obvod}
Pro nabíjecí obvod byl zvolen čip TP4056. Jeho zapojení bylo převzato z datasheetu [citace]. %tady bude obrázek schématu asi
Velikost rezistoru $R_\textind{PROG}$ se volí podle nabíjecího proudu. 
Tabulka rezistorů $R_\textind{PROG}$.. [citace]

\begin{table}[!h]
    \caption{Nastavení nabíjecího proudu rezitorem $R_\textind{PROG}$}
    \begin{center}
        \begin{tabular}{|c|c|}
            \hline
            $R_\textind{PROG}$ [kOhm] & Nabíjecí proud [mA] \\
            \hline
            10      & 130 \\
            \hline
            5       & 250 \\
            \hline
            4       & 300 \\
            \hline
            3       & 400 \\
            \hline
            2       & 580 \\
            \hline
            1,66    & 690 \\
            \hline
            1,5     & 780 \\
            \hline
            1,33    & 900 \\
            \hline
            1,2     & 1000 \\
            \hline
        \end{tabular}
        
    \end{center}
\end{table}

Nabíjení baterií by mělo probíhat při 0,5C, tudíž pro 18650 je to cca 0,5 A. Proto byl zvolen rezitor Rprog 2 kOhm.

\section{Měření baterie}

\section{Stepdown}

\section{Ochrana proti přepólování}

\section{Zapínání LED}

\section{Programovací převodník}
Jako nabíjecí a zároveň programovací konektor slouží USB. Procesor má pouze programovací rozhraní RS-232. Aby mohlo programování probíhat z USB, 
musí DPS obsahovat převodník z USB na RS-232. Byl zvolen převodník CP2102. 

Byl zvolen protože na ESP32 devkitu C je použit právě tento převodník. Byl zapojen podle doporučeného zapojení v datasheetu [citace].

\chapter{Ovládání hry}