% Soubory musí být v kódování, které je nastaveno v příkazu \usepackage[...]{inputenc}

\documentclass[%        Základní nastavení
  %draft,    				  % Testovací překlad
  12pt,       				% Velikost základního písma je 12 bodů
	t,                  % obsah slajdů bude vždy začínat od shora (nebude vertikálně centrovaný)
	aspectratio=1610,   % poměr stran bude 16:10 (všechny projektory v učebnách na Technické 12 Brno),
	                    % další volby jsou 43, 149, 169, 54, 32.
	unicode,						% Záložky a informace budou v kódování unicode
]{beamer}				    	% Dokument třídy 'zpráva', vhodná pro sazbu závěrečných prací s kapitolami
%\usepackage{etex}

\usepackage[utf8]		  % Kódování zdrojových souborů je v UTF-8
	{inputenc}					% Balíček pro nastavení kódování zdrojových souborů
	
\usepackage{graphicx} % Balíček 'graphicx' pro vkládání obrázků
											% Nutné pro vložení logotypů školy a fakulty

\usepackage[          % Balíček 'acronym' pro sazby zkratek a symbolů
	nohyperlinks				% Nebudou tvořeny hypertextové odkazy do seznamu zkratek
]{acronym}						
											% Nutné pro použití prostředí 'acronym' balíčku 'thesis'

%% Balíček hyperref je volán třídou beamer automaticky, proto není třeba následujícího kódu:
%\usepackage[
%	breaklinks=true,		% Hypertextové odkazy mohou obsahovat zalomení řádku
%	hypertexnames=false % Názvy hypertextových odkazů budou tvořeny
%											% nezávisle na názvech TeXu
%]{hyperref}						% Balíček 'hyperref' pro sazbu hypertextových odkazů
%											% Nutné pro použití příkazu 'nastavenipdf' balíčku 'thesis'

\usepackage{cmap} 		% Balíček cmap zajišťuje, že PDF vytvořené `pdflatexem' je
											% plně "prohledávatelné" a "kopírovatelné"

%\usepackage{upgreek}	% Balíček pro sazbu stojatých řeckých písmem
											%% např. stojaté pí: \uppi
											%% např. stojaté mí: \upmu (použitelné třeba v mikrometrech)
											%% pozor, grafická nekompatibilita s fonty typu Computer Modern!

%\usepackage{amsmath} %balíček pro sabu náročnější matematiky

\usepackage{booktabs} % Balíček, který umožňuje v tabulce používat
                      % příkazy \toprule, \midrule, \bottomrule


%%%%%%%%%%%%%%%%%%%%%%%%%%%%%%%%%%%%%%%%%%%%%%%%%%%%%%%%%%%%%%%%%
%%%%%%      Definice informací o dokumentu             %%%%%%%%%%
%%%%%%%%%%%%%%%%%%%%%%%%%%%%%%%%%%%%%%%%%%%%%%%%%%%%%%%%%%%%%%%%%

\input{nastaveni}      % v tomto souboru doplňte údaje o sobě, o názvu práce...
                       % (tento soubor je sdílený s textem práce)

%%%%%%%%%%%%%%%%%%%%%%%%%%%%%%%%%%%%%%%%%%%%%%%%%%%%%%%%%%%%%%%%%%%%%%%%

%%%%%%%%%%%%%%%%%%%%%%%%%%%%%%%%%%%%%%%%%%%%%%%%%%%%%%%%%%%%%%%%%%%%%%%%
%%%%%%     Nastavení polí ve Vlastnostech dokumentu PDF      %%%%%%%%%%%
%%%%%%%%%%%%%%%%%%%%%%%%%%%%%%%%%%%%%%%%%%%%%%%%%%%%%%%%%%%%%%%%%%%%%%%%
%% Při vloženém balíčku 'hyperref' lze použít příkaz '\pdfsettings'
\pdfsettings
%  Nastavení polí je možné provést také ručně příkazem:
%\hypersetup{
%  pdftitle={Název studentské práce},    	% Pole 'Document Title'
%  pdfauthor={Autor studenstké práce},   	% Pole 'Author'
%  pdfsubject={Typ práce}, 						  	% Pole 'Subject'
%  pdfkeywords={Klíčová slova}           	% Pole 'Keywords'
%}
\hypersetup{pdfpagemode=FullScreen}       % otevření rovnou v režimu celé obrazovky
%%%%%%%%%%%%%%%%%%%%%%%%%%%%%%%%%%%%%%%%%%%%%%%%%%%%%%%%%%%%%%%%%%%%%%%

\usetheme{VUT} 				% barvy a rozložení prezentace odpovídající VUT FEKT
% alternativně lze použít jiná berevná témata, ale bez záruky. Například: 
%\usetheme{Darmstadt} \usecolortheme{default2}
\logoheader					% vytvoření zkráceného loga VUT FEKT v hlavičce slajdu, nechte odkomentované



\begin{document}

% v případě zakomentování následujícího se zobrazí v pravém dolním rohu slajdů klikatelné navigační symboly 
\disablenavigationsymbols

% titulní snímek, vysazen bez horních, dolních a postranních lišt (volba plain),
% není tak vysazen ani nadpis snímku
\maketitle

%%%%%%%%%%%%%%%%%%%%%%%%%%%%%%%%%%%%%%%%%%%%%%%%%%%%%%%%%%%%%%%%%%%%%%%
% 1. snímek s cíli (zadaním) práce
\begin{frame} 
	% nadpis snímku
	\frametitle{Cíle práce}
	\vspace{1cm}
	\begin{itemize}
			\item Nastudování pravidla
			\item Návrh
			\item Výroba
			\item Testování
			\item Ovládání hry
	\end{itemize}
\end{frame}


%%%%%%%%%%%%%
\begin{frame} 
	\frametitle{Pravidla}
	\begin{columns}[T] 								% prostředí sloupce s umístěním nahoře
		\begin{column}{0.5\textwidth}		% první sloupec
			\vspace{0.5cm}
			\begin{itemize}
				\item Respektování pravidel deskové hry
				\item Zadání kombinace barev v určitém pořadí
				\item Vyhodnocení
				\begin{itemize}
					\item Správná barva na správné pozici
					\item Správná barva na špatné pozici
				\end{itemize}
				\item 10 pokusů
			\end{itemize}
		\end{column}
		%
		\begin{column}{0.5\textwidth}		% druhý sloupec
			\begin{figure}%	
				\centering
				\vspace{0.5cm}	              % horizontální mezera
				\includegraphics[width=1.1\columnwidth]{obrazky/Logic_deskovka.png}
				%lze vložit popisek, ale povetšinou je to v prezentaci zbytečné
				%\caption{Popisek obrázku}%
				%\label{obr:ukazka}
			\end{figure}
		\end{column}
	\end{columns}	
\end{frame} 


	% prostředí 'alertblock', které slouží pro zdůraznění informace
	%\begin{alertblock}{Pro práci je klíčový Eulerův vzorec}
	%	$$\eul^{\jmag x}=\cos x + \jmag\sin x$$
	%\end{alertblock}

	%\vspace{4ex}
	%Eulerova identita je speciálním případem tohoto vzorce, jestliže dosadíme $x=\uppi$\,:

	% prostředí 'block', které slouží jako informativní
	%\begin{block}{Eulerova identita}
	%	$$\eul^{\jmag \uppi}=\cos \uppi + \jmag\sin \uppi,$$\\
	%	odkud vyplývá
	%	$$\eul^{\jmag \uppi}+1=0.$$
	%\end{block}
%%%%%%%%%%%%%
\begin{frame} 
	\frametitle{Návrh elektroniky}
	
	\begin{columns}[T] 								% prostředí sloupce s umístěním nahoře
		\begin{column}{0.45\textwidth}		% první sloupec
			\vspace{0.5cm}
			\begin{itemize}
				\item Řídicí elektronika
				\item Napájení
				\item Měnič napětí
				\item Převodník z USB na RS-232
				\item Herní prvky
				\item Ovládací prvky
			\end{itemize}
		\end{column}
		%
		\begin{column}{0.55\textwidth}		% druhý sloupec
			\begin{figure}%	
				\centering
				\vspace{0.5cm}	              % horizontální mezera
				\includegraphics[width=1\columnwidth]{obrazky/v1_blokove_schema.jpg}
				%lze vložit popisek, ale povetšinou je to v prezentaci zbytečné
				%\caption{Popisek obrázku}%
				%\label{obr:ukazka}
			\end{figure}
		\end{column}
	\end{columns}											% ukončení prostředí sloupce
\end{frame}

\begin{frame} 
	\frametitle{Herní prvky}
	\begin{columns}[T] 								% prostředí sloupce s umístěním nahoře
		\begin{column}{0.4\textwidth}		% první sloupec
			%Obrázek znázorňuje model:\\[2ex]
			\vspace{0.5cm}
			\begin{itemize}
				\item WS2812C
				\item RGB
				\item Napájení
				\item Princip spojení
				\item Komunikace
			\end{itemize}
		\end{column}
		%
		\begin{column}{0.7\textwidth}		% druhý sloupec
			\begin{figure}%	
				\centering
				%\vspace{0.5cm}	              % horizontální mezera
				\includegraphics[width=0.3\columnwidth]{obrazky/WS2812C.jpg}
				%lze vložit popisek, ale povetšinou je to v prezentaci zbytečné
				%\caption{Popisek obrázku}%
				%\label{obr:ukazka}
			\end{figure}
		\end{column}
	\end{columns}	
	
	\begin{figure}%	
		\centering
		\vspace{0.5cm}	              % horizontální mezera
		\includegraphics[width=1\columnwidth]{obrazky/WS2812C_spojeni.png}
	\end{figure}% ukončení prostředí sloupce
\end{frame}


%%%%%%%%%%%%%
\begin{frame} 
	\frametitle{Návrh DPS}
		\begin{columns}[T] 	
		\begin{column}{0.4\textwidth}		% první sloupec
			%Obrázek znázorňuje model:\\[2ex]
			\vspace{0.5cm}
			\begin{itemize}
				\item KiCad
				\item JLCPCB
				\item 4 vrstvy
			\end{itemize}
			\vspace{0.5cm}
			\textbf{Vzhled}
			\begin{itemize}
				\item 10 ~$\times$ 10 cm
				\item 3 části inteligentních LED
			\end{itemize}
		\end{column}

		\begin{column}{0.6\textwidth}		% druhý sloupec
			\begin{figure}%	
				\centering
				%\vspace{0.5cm}	              % horizontální mezera
				\includegraphics[width=0.8\columnwidth]{prilohy/Verze1_DPS_cela.png}
			\end{figure}% ukončení prostředí sloupce
		\end{column}
	\end{columns}
\end{frame}

\begin{frame} 
	\frametitle{Návrh DPS}
		\begin{columns}[T] 	
		\begin{column}{0.5\textwidth}		% druhý sloupec
			\vspace{0.5cm}
			\textbf{Funkční rozmístění součástek}
			\begin{itemize}
				\item Kondenzátory co nejblíže pouzdra součáskty
				\item D+ a D- diferenciální pár
				\item Rozložení součástek měniče podle dokumentace
			\end{itemize}
		\end{column}
		\begin{column}{0.5\textwidth}		% druhý sloupec
			\begin{figure}%	
				\centering
				%\vspace{0.5cm}	              % horizontální mezera
				\includegraphics[width=1\columnwidth]{obrazky/SY8105_rozlozeni_na_DPS.png}
			\end{figure}% ukončení prostředí sloupce
		\end{column}
	\end{columns}
\end{frame}

	%\begin{table}[]
	%	\centering
	%	\caption{Výsledky měření mobilních sítí}
		%\label{tab:tabulka}
		%	\begin{tabular}{lcc}
		%		\toprule
		%			Technologie  & Rychlost stahování [kB/s] & Rychlost nahrávání [kB/s] \\
		%		\midrule
		%			GPRS (2,5G)	& 7,2 	& 3,6\\
		%			UMTS 3G     & 48 		& 48\\
		%			HSPA (3,5G)	&	1\,706	&	720\\
		%			LTE (4G) 		& 40\,750 & 10\,750\\
		%		\bottomrule                                       
		%	\end{tabular}
	%\end{table}

\begin{frame} 
	\frametitle{Oživení}
	\begin{columns}[T] 	
		\begin{column}{0.4\textwidth}		% druhý sloupec
			\vspace{0.5cm}
			\begin{itemize}
				\item Osazení THT komponent
				\item Vložení baterií do pouzdra
				\item LED indikující napájecí napětí
				\item Chybná pouzdra tranzistorů
				\item Tlačítko připojeno na špatném pinu
			\end{itemize}
		\end{column}
		\begin{column}{0.6\textwidth}		% druhý sloupec
			\begin{figure}%	
				\centering
				\vspace{0.3cm}	              % horizontální mezera
				\includegraphics[width=0.8\columnwidth]{obrazky/Verze0_zapnuto_nabito.jpg}
			\end{figure}% ukončení prostředí sloupce
		\end{column}
	\end{columns}
	
\end{frame}

\begin{frame} 
	\frametitle{Způsob hry}
	\begin{figure}%	
		\centering
		%\vspace{0.5cm}	              % horizontální mezera
		\includegraphics[width=0.5\columnwidth]{obrazky/vyvojovy_diagram.png}
	\end{figure}% ukončení prostředí sloupce
	%\begin{table}[]
	%	\centering
	%	\caption{Výsledky měření mobilních sítí}
	%	\label{tab:tabulka}
	%		\begin{tabular}{lcc}
	%			\toprule
	%				Technologie  & Rychlost stahování [kB/s] & Rychlost nahrávání [kB/s] \\
	%			\midrule
	%				GPRS (2,5G)	& 7,2 	& 3,6\\
	%				UMTS 3G     & 48 		& 48\\
	%				HSPA (3,5G)	&	1\,706	&	720\\
	%				LTE (4G) 		& 40\,750 & 10\,750\\
	%			\bottomrule                                       
	%		\end{tabular}
	%\end{table}
\end{frame}

\begin{frame} 
	\frametitle{Plány do budoucna}
	\vspace{0.5cm}
	\begin{columns}[T] 								% prostředí sloupce s umístěním nahoře
		\begin{column}{0.4\textwidth}		% první sloupec
			\begin{itemize}
				\item Software
				\item Konfigurace pro 2 hráče
				\item Návrh a výroba krabičky
				\item Jiné způsoby ovládání
				\item Snížení spotřeby - závislost jasu inteligentních LED na okolním osvětlení
			\end{itemize}
		\end{column}
		%
		\begin{column}{0.6\textwidth}		% druhý sloupec
		\end{column}
	\end{columns}	
\end{frame}

%%%%%%%%%%%%%
\begin{frame} 
	\frametitle{Závěr}
	\begin{columns}[T] % prostředí sloupce s umístěním nahoře								
		\begin{column}{0.3\textwidth}		% první sloupec
			\vspace{0.5cm}
			\begin{itemize}
				\item Pravidla
				\item Návrh elektroniky
				\item Návrh DPS
				\item Oživení
				\item Způsob hry
			\end{itemize}
		\end{column}
		%
		\begin{column}{0.7\textwidth}		% druhý sloupec
			\begin{figure}%	
				\centering
				%\vspace{0.1cm}	              % horizontální mezera
				\includegraphics[width=0.7\columnwidth]{obrazky/Verze1_3D_pohled.png}
				%lze vložit popisek, ale povetšinou je to v prezentaci zbytečné
				%\caption{Popisek obrázku}%
				%\label{obr:ukazka}
			\end{figure}
		\end{column}
	\end{columns}	
\end{frame}


% podekovani
\begin{frame}[c] 
% bez nadpisu snímku
	\frametitle{\mbox{ }}
	\begin{center}
		{\Huge Děkuji za pozornost!}
	\end{center}
\end{frame}

% otázky oponenta
%\frame{
%\frametitle{Otázky oponenta}
%	\emph{Jaká je souvislost Vašeho vzorce (1.2) s~Maxwellovými rovnicemi v~integrálním tvaru?}\\[2ex]
	%
%	Již staří Římané\,\dots
%}

\end{document}
